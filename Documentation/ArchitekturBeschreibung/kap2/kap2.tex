\textbf{Autor: Marvin}\\
{Es werden zunächst Einflussfaktoren für das Projekt identifiziert und analysiert. Danach werden unter Einbeziehung der Einflussfaktoren Problemkarten mitsamt Lösungsstrategien definiert. Jeder Einflussfaktor muss für die Architektur relevant sein.}
\subsection{Einflussfaktoren} \label{sec:einflussfaktoren}
{Die Einflussfaktoren werden eingeteilt in organisatorische, technische oder produktspezifische Faktoren, um eine bessere Übersichtq über die Einflussfaktoren zu erhalten.\\
Jeder Einflussfaktor wird charakterisiert in seiner Flexibilität und Änderbarkeit.\\
Flexibilität: Ist der Faktor zum jetzigen Zeitpunkt verhandelbar?\\
Änderbarkeit: Könnte sich der Faktor durch äußere Einflüsse innerhalb des Projektzeitraums ändern?\\
Anschließend werden die Auswirkungen des Einflussfaktors auf das Projekt definiert unter Einbeziehung der Charakteristiken.

\subsubsection{Organisatorische Faktoren}

\textbf{O1 Entwicklungsplanung}

% Die erste Tabelle
\begin{table}[htbp]
\centering
\begin{tabular}{|l|p{0.8\linewidth}|}
\hline
\multicolumn{2}{|l|}{\textbf{O1.1 Time-To-Market}}
  \tabularnewline \hline
Beschreibung                                               &  Die Software muss spätestens bis zum 11.03.2018 ausgeliefert worden sein.                                                   \tabularnewline \hline
Flexibilität                                              & Der Termin ist nicht verhandelbar.                                                                                \tabularnewline \hline
Änderbarkeit                                         & Der Termin wird nur durch den Kunden bestimmt und der Kunde wird den Termin  nicht ändern.                                                                 \tabularnewline \hline
Auswirkungen                                                & Die Mindestanforderungen müssen bis zu diesem Termin vollständig umgesetzt sein. Die Frist gilt ebenso für die Umsetzung von optionalen Anforderungen.                                         \tabularnewline \hline
 \hline
\end{tabular}
\end{table}

% Die zweite Tabelle

\begin{table}[htbp]
\centering
\begin{tabular}{|l|p{0.8\linewidth}|}
\hline
\multicolumn{2}{|l|}{\textbf{O1.2 Projektbudget}}
  \tabularnewline \hline
Beschreibung                                               & Das Projektbudget beträgt 0 Euro.                                      \tabularnewline \hline
Flexibilität                                              & Das Budget ist nicht verhandelbar.                                                                                \tabularnewline \hline
Änderbarkeit                                         & Das Budget wird sich nicht ändern.                                                                 \tabularnewline \hline
Auswirkungen                                                & Die Umsetzung der Anforderungen muss mit OpenSource Produkten erfolgen, da keine Mittel für Lizensierungen zur Verfügung stehen.                                       \tabularnewline \hline
 \hline
\end{tabular}
\end{table}
% die dritte Tabelle
\begin{table}[htbp]
\centering
\begin{tabular}{|l|p{0.8\linewidth}|}
\hline
\multicolumn{2}{|l|}{\textbf{O1.3 Team-Kapazität}}
  \tabularnewline \hline
Beschreibung                                               & Das Team besteht aus sechs Entwicklern.                                      \tabularnewline \hline
Flexibilität                                              & Die Größe des Teams ist nicht verhandelbar. \tabularnewline \hline
Änderbarkeit                                         & Die Teamgröße darf nicht aufgestockt werden, jedoch können Teammitglieder das Entwicklerteam verlassen. \tabularnewline \hline
Auswirkungen                                                & Die Umsetzung der Anforderungen muss mit der gegebenen Teamgröße umgesetzt werden.                                      \tabularnewline \hline
 \hline
\end{tabular}
\end{table}
\newpage
\textbf{O2 Gruppenkonstellation}

% Die vierte Tabelle
\begin{table}[htbp]
\centering
\begin{tabular}{|l|p{0.8\linewidth}|}
\hline
\multicolumn{2}{|l|}{\textbf{O2.1 Zielsetzung }}
  \tabularnewline \hline
Beschreibung                                               &  Das Ziel des Projektteam ist das Erreichen einer sehr guten Bewertung.                                                 \tabularnewline \hline
Flexibilität                                              & Das angestrebte Ziel ist gruppenintern verhandelbar.                                                                               \tabularnewline \hline
Änderbarkeit                                         & Das angestrebte Ziel könnte sich ändern, falls Druck durch andere Lehrveranstaltungen entsteht.                                                               \tabularnewline \hline
Auswirkungen                                                & Die Implementierung der Kundenanforderungen geht über die Mindestanforderungen hinaus. Der Kunde soll zufriedengestellt werden.                                    \tabularnewline \hline
 \hline
\end{tabular}
\end{table}

% Die fünfte Tabelle


\centering
\begin{tabular}{|l|p{0.8\linewidth}|}
\hline
\multicolumn{2}{|l|}{\textbf{O2.2 Geringe Erfahrung}}
  \tabularnewline \hline
Beschreibung                                               &  Das Projektteam ist unerfahren in der Softwareentwicklung.                                                  \tabularnewline \hline
Flexibilität                                              & Die Erfahrung des Projektteams steht zum Projektbeginn fest.                                                                                \tabularnewline \hline
Änderbarkeit                                         & Die Erfahrung nimmt im Verlauf des Softwareprojekts zu.                                                               \tabularnewline \hline
Auswirkungen                                                & Mangelnde Erfahrung kann zu Fehlern und falschen Designentscheidungen führen.                                       \tabularnewline \hline
 \hline
\end{tabular}


\newpage

% Nächster Faktorabschnitt
\flushleft
\subsubsection{Produktfaktoren}

\textbf{P1 Vorgaben des Kunden} 
% Tabelle 1

\centering
\begin{tabular}{|l|p{0.8\linewidth}|}
\hline
\multicolumn{2}{|l|}{\textbf{P1.1 Einbindung des Google-Kalendars}}
  \tabularnewline \hline
Beschreibung                                               &  Die Software muss sich mit den Terminen des Google Kalendars synchronisieren. \tabularnewline \hline
Flexibilität                                              & Die Einbindung ist optional und unterliegt der Entscheidung des Projektteams. \tabularnewline \hline
Änderbarkeit                                         & Die Einbindung kann gestrichen werden, falls die Zeit knapp wird oder schwerwiegende Probleme auftreten. Es wird nur die Einbindung des Google Kalendars erwartet. Wobei noch weitere Kalender später hinzukommen könnten.                                                               \tabularnewline \hline
Auswirkungen                                                & Die Software muss eine Kalender Schnittstelle bereitstellen zur Snchronisation der Termine in der Software. \tabularnewline \hline
 \hline
\end{tabular}

% Tabelle 2

\centering
\begin{tabular}{|l|p{0.8\linewidth}|}
\hline
\multicolumn{2}{|l|}{\textbf{P1.2 User-Verwaltung}}
  \tabularnewline \hline
Beschreibung                                               &  Es werden verschiedene Rollen (Prüfling, Sektetär, Prüfer, Administrator) in der Software abgebildet. \tabularnewline \hline
Flexibilität                                              & Die Rolle des Sekretärs ist verhandelbar. Die anderen Rollen müssen implementiert werden.                                                                                \tabularnewline \hline
Änderbarkeit                                         & Der Einbindung des Sekretärs wird verworfen, falls sich Probleme in der Implementierung ergeben oder falls die Zeit knap wird. Die anderen Rollen werden sich nicht ändern.                                                                 \tabularnewline \hline
Auswirkungen                                                & Die Rollen (Prüfling, Prüfer, Administrator) müssen in der Software eingebunden werden und das mit ihren Funktionen und Charakteristiken.  \tabularnewline \hline
 \hline
\end{tabular}

% Tabelle 3

\centering
\begin{tabular}{|l|p{0.8\linewidth}|}
\hline
\multicolumn{2}{|l|}{\textbf{P1.3 Mehrsprachigkeit}}
  \tabularnewline \hline
Beschreibung                                               &  Es sollen Englisch und Deutsch unterstüzt werden von der Software. \tabularnewline \hline
Flexibilität                                              & Die Sprachen sind nicht verhandelbar.                                                                                \tabularnewline \hline
Änderbarkeit                                         & Weitere Sprachen sollten später hinzufügbar sein.                                                                 \tabularnewline \hline
Auswirkungen                                                & Die Software muss verschiedene Sprachen unterstützen und stellt dafür eine Schnittstelle bereit.                                         \tabularnewline \hline
 \hline
\end{tabular}


% Tabelle 4

\centering
\begin{tabular}{|l|p{0.8\linewidth}|}
\hline
\multicolumn{2}{|l|}{\textbf{P1.4 Applikation für mobile Geräte}}
  \tabularnewline \hline
Beschreibung                                               &  Es soll eine mobile Anwendung für iOS und Android Geräte bereitgestellt werden. \tabularnewline \hline
Flexibilität                                              & Der Umsetzung in Form einer App oder als Responsive-Website ist verhandelbar.                                                                                \tabularnewline \hline
Änderbarkeit                                         & Die Software soll mobil benutzbar sein.                                                                 \tabularnewline \hline
Auswirkungen                                                & Die Umsetzung einer App erfordert Wissen über Android und iOS und das Design mitsamt der Funktionen müssen für die kleineren Bildschirme angepasst werden. \tabularnewline \hline
 \hline
\end{tabular}

% Tabelle 4.1

\centering
\begin{tabular}{|l|p{0.8\linewidth}|}
\hline
\multicolumn{2}{|l|}{\textbf{P1.5 Termin-Verwaltung}}
  \tabularnewline \hline
Beschreibung                                               &  Der Prüfer trägt seine Termine in den Kalender ein. Bei Konflikten werden Lösungen angeboten. Eine Termineintragung unterscheidet zwischen Einzel- und Gruppeneintragungen.  \tabularnewline \hline
Flexibilität                                              & Die Funktion ist nicht verhandelbar. \tabularnewline \hline
Änderbarkeit                                         & Die Funktion wird sich nicht ändern im Produktlebenszyklus.                                                      \tabularnewline \hline
Auswirkungen                                                &  Die Termine müssen beschafft werden. Es müssen im Konfliktfall per Algorithmus neue Termine gefunden werden und an alle Termin-Teilnehmer verteilt.  \tabularnewline \hline
 \hline
\end{tabular}

% Tabelle 4.2
\centering
\begin{tabular}{|l|p{0.8\linewidth}|}
\hline
\multicolumn{2}{|l|}{\textbf{P1.6 Augabe von Dokumenten im PDF-Format}}
  \tabularnewline \hline
Beschreibung                                               &  Der Prüfer soll sich Protokoll und Quittung in Form von PDF-Dokumenten erstellen können. \tabularnewline \hline
Flexibilität                                              & Die Funktion ist nicht verhandelbar. \tabularnewline \hline
Änderbarkeit                                         & Falls sich die Prüfungsregularien ändern, könnte auch die nötigen PDF-Dokumente sich ändern.                                                               \tabularnewline \hline
Auswirkungen                                                & Es muss das Schreiben von Protokollen und Quittungen zur Verfügung gestellt werden. Die Ergebnisse daraus müss als PDF exportiert und importiert werden können \tabularnewline \hline
 \hline
\end{tabular}

\raggedright
\textbf{P2 Nutzbarkeit}
% Tabelle 5

\centering
\begin{tabular}{|l|p{0.8\linewidth}|}
\hline
\multicolumn{2}{|l|}{\textbf{P2.1 Usability}}
  \tabularnewline \hline
Beschreibung                                               &  Die Software soll eine sehr gute Usability bieten. \tabularnewline \hline
Flexibilität                                              & Die Usability ist nicht verhandelbar.                                                                                \tabularnewline \hline
Änderbarkeit                                         & Die Usability wird sich an den Nutzern ausrichten.                                                                 \tabularnewline \hline
Auswirkungen                                                & Es werden unterschiedliche Benutzergruppen angesprochen. Möglicherweise werden unterschiedliche User-Interfaces für unterschiedliche Nutzergruppen bereitgestellt. Die Software soll fehlertolerant, lernförderlich, effektiv und effizient sein. Dabei Sicherheit und Freude bieten \tabularnewline \hline
 \hline
\end{tabular}


% Nächster Faktorabschntt
\subsubsection{Technische Faktoren}
\raggedright
\textbf{T1 Unterstützte Systeme}
% Tabelle 1

\centering
\begin{tabular}{|l|p{0.8\linewidth}|}
\hline
\multicolumn{2}{|l|}{\textbf{T1.1 Unterstützte Browser}}
  \tabularnewline \hline
Beschreibung                                               &  Die Software muss gängie Browser unterstützen. Darunter zählen Internet Explorer 11, Microsoft Edge, Firefox, Chrome und Safari.                                                 \tabularnewline \hline
Flexibilität                                              & Die Auswahl der Browser ist nicht verhandelbar.                                                                                \tabularnewline \hline
Änderbarkeit                                         & Es werden nur bestimmte Versionen der Browser unterstützt.                                                        \tabularnewline \hline
Auswirkungen                                                & Das Design der Software muss auf allen genannten Browsern Kundenzufriedenheit auslösen, obwohl die Browser Eigenheiten aufweisen.                              \tabularnewline \hline
 \hline
\end{tabular}

% Tabelle 2
\centering
\begin{tabular}{|l|p{0.8\linewidth}|}
\hline
\multicolumn{2}{|l|}{\textbf{T1.2 Unterstützte Betriebssysteme}}
  \tabularnewline \hline
Beschreibung                                               &  Die Software muss auf den Betriebssystemen Windows 10, Linux und McOS laufen.                                               \tabularnewline \hline
Flexibilität                                              & Die Auswahl der Betriebssysteme ist nicht verhandelbar.                                                                                \tabularnewline \hline
Änderbarkeit                                         & Es werden nur bestimmte Versionen der Betriebssysteme unterstützt.                                                        \tabularnewline \hline
Auswirkungen                                                & Die Software muss auf den genannten Betriebssystemen identisch laufen, obwohl die Betriebssysteme ihre Eigenheiten aufweisen.                                    \tabularnewline \hline
 \hline
\end{tabular}

\begin{flushleft}
\textbf{T2  Datenbank}
\end{flushleft}
% Tabelle 3
\centering
\begin{tabular}{|l|p{0.8\linewidth}|}
\hline
\multicolumn{2}{|l|}{\textbf{T2.1 Einbindung einer Datenbank}}
  \tabularnewline \hline
Beschreibung                                               &  Die Datenbank soll mit GlassFish laufen.                                               \tabularnewline \hline
Flexibilität                                              & Die Auswahl der Datenbank ist nicht verhandelbar.                                                                                \tabularnewline \hline
Änderbarkeit                                         &  Die Datenbank muss austauschbar sein, falls die Unterstützung ausläuft oder die Datenbank zu langsam wird. \tabularnewline \hline
Auswirkungen                                                & Die Datenbank muss über eine Schnittstelle angebunden werden, um das Austauschen der Datenbank zu ermöglichen.               \tabularnewline \hline
 \hline
\end{tabular}

% Tabelle 4
\centering
\begin{tabular}{|l|p{0.8\linewidth}|}
\hline
\multicolumn{2}{|l|}{\textbf{T2.2 Datenschutz}}
  \tabularnewline \hline
Beschreibung                                               &  Die Daten der Benutzer dürfen nur im Rahmen des Datenschutzes genutzt werden.                                                 \tabularnewline \hline
Flexibilität                                              & Die Einhaltung des Datenschutzes ist nicht verhandelbar. \tabularnewline \hline
Änderbarkeit                                         & Es könnten neue Datenschutzregelungen in Kraft treten. \tabularnewline \hline
Auswirkungen                                                & Das System arbeitet nur mit dringend benötigten Daten. Es gilt das Prinzip der Datensparsamkeit. Die Daten werden annonymisiert.                                  \tabularnewline \hline
 \hline
\end{tabular}

% Tabelle 5
\centering
\begin{tabular}{|l|p{0.8\linewidth}|}
\hline
\multicolumn{2}{|l|}{\textbf{T2.3 Datensicherheit}}
  \tabularnewline \hline
Beschreibung                                               &  Die Daten der Benutzer müssen gesichert werden vor böswilligen Zugriffen.                                                 \tabularnewline \hline
Flexibilität                                              & Die Datensicherheit ist nicht verhandelbar.                                                                                \tabularnewline \hline
Änderbarkeit                                         & Die Datensicherheit muss sich an neue Bedrohungen und Entwicklungen anpassen.                                                        \tabularnewline \hline
Auswirkungen                                                & Die Software muss die Daten des Nutzers vor den Zugriff von Dritten schützen. Dazu werden Sicherheitsmaßnahmen benötigt. \tabularnewline \hline
 \hline
\end{tabular}

% Tabelle 6
\centering
\begin{tabular}{|l|p{0.8\linewidth}|}
\hline
\multicolumn{2}{|l|}{\textbf{T2.4 Daten-Backups}}
  \tabularnewline \hline
Beschreibung                                               &  Der Datenbestand soll gesichert werden können.                                                \tabularnewline \hline
Flexibilität                                              & Das Daten-Backup ist nicht verhandelbar.                                                                                \tabularnewline \hline
Änderbarkeit                                         & Das Daten-Backup wiederfährt keinen Änderungen.                                                        \tabularnewline \hline
Auswirkungen                                                & Der gesamte Datenbestand muss ein- und auslesbar sein. Diese Funktion ist ein Sicherheitsrisiko. Die Datenbestand soll gespeichert werden können . Der Datenbestand soll überschreibbar sein.                               \tabularnewline \hline
 \hline
\end{tabular}
\newpage
\begin{flushleft}
\textbf{T3  Technologievorgaben}
\end{flushleft}
% Tabelle 7
\centering
\begin{tabular}{|l|p{0.8\linewidth}|}
\hline
\multicolumn{2}{|l|}{\textbf{T3.1 Programmiersprache}}
  \tabularnewline \hline
Beschreibung                                               &  Die genutzte Sprache ist Java 7 oder höher.                                                \tabularnewline \hline
Flexibilität                                              & Die Programmiersprache ist nicht verhandelbar.                                                                                \tabularnewline \hline
Änderbarkeit                                         & Es werden für die genutzten Programmiersprache  Updates bereitgestellt.                                                        \tabularnewline \hline
Auswirkungen                                                & Das Programm ist auf Bibliotheken und Features von Java beschränkt.                                \tabularnewline \hline
 \hline
\end{tabular}

% Tabelle 8
\centering
\begin{tabular}{|l|p{0.8\linewidth}|}
\hline
\multicolumn{2}{|l|}{\textbf{T3.2 Client-Server-Struktur}}
  \tabularnewline \hline
Beschreibung                                               &  Der Nutzer soll über einen Client auf einen Server zugreifen. Die Implentierung des Servers soll mit Java Server Faces erfolgen.                                       \tabularnewline \hline
Flexibilität                                              & Die Implementierung des Servers ist nicht verhandelbar.                                                                                \tabularnewline \hline
Änderbarkeit                                         & Es wird eine Java Server Faces Version ausgewählt. Die Client-Server-Struktur wird sich nicht ändern.                                                        \tabularnewline \hline
Auswirkungen                                                &  Durch die Nutzung von Java Server Faces ist die Client-Server-Struktur auf Funktionen von Java Server Faces beschränkt.                                 \tabularnewline \hline
 \hline
\end{tabular}

\begin{flushleft}
\textbf{T4  Systemeigenschaften}
\end{flushleft}
% Tabelle 9
\centering
\begin{tabular}{|l|p{0.8\linewidth}|}
\hline
\multicolumn{2}{|l|}{\textbf{T4.1 Wartbarkeit}}
  \tabularnewline \hline
Beschreibung                                               &  Das System soll wart- und erweiterbar sein. \tabularnewline \hline
Flexibilität                                              & Die Wartbarkeit ist verhandelbar, da sie nicht explizit gewünscht wurde.                                                                                \tabularnewline \hline
Änderbarkeit                                         & Die Software wird mit jeder geschriebenen Codezeile schwieriger zu warten.                                                       \tabularnewline \hline
Auswirkungen                                                & Die Software soll eine hohe Kohärenz und wenig Abhängigkeiten aufweisen. Die Nutzung von Interfaces zur Abstraktion von Implementierungen sollen ebenfalls helfen.                                \tabularnewline \hline
 \hline
\end{tabular}


% Die Probleme und Strategien
\subsection{Probleme und Strategien} \label{sec:strategien}
Die Menge der Einflusssfaktoren werden auf mögliche Probleme hin global analysiert. Die gefundenen Probleme werden  auf Problemkarten (K) festgehalten. Eine Problemkarte enthält eine Beschreibung des Problems, die betreffenden Faktoren, mehrere Lösungsstrategien, sowie die ausgewählte Lösungsstrategie für das Problem.

% Strategie 1
\begin{center}
\begin{minipage}{\linewidth}
    \centering
\renewcommand{\arraystretch}{1.5}
\begin{tabular}{|>{\centering\arraybackslash}p{15cm}|}
            \hline
            \label{K1}
           \textbf{K1 Systemaufbau}\\ \hline
            \textit{Der Benutzer greift mit seinem Browser auf das System zu. Die Daten der Benutzer sollen gespeichert werden. Die Benutzer können Daten manipulieren und mit dem System interagieren.}\\ \hline
            \textit{Betroffene Faktoren:} \\ 
	O1.1 Time-To-Market, O1.3 Team-Kapazität, O2.2 Geringe Erfahrung, T3.1 Programmiersprache, T3.2 Client-Server-Struktur, T4.1 Wartbarkeit
	\\ \hline
	      \textit{Strategien:}\\ 
\textbf{S1}: Es wird das Model-View-Controller Entwurfsmuster verwendet für die Softwarearchitektur. Die Schnittstellen der einzelenen Module sind darin genau geregelt. Die Wartbarkeit und Lesbarkeit würde von der Modularisierung profitieren.\\
\textbf{S2}: Die Softwarearchitektur besteht aus einem großen Monolithen. Klassen und Methoden wären im gesamten Monolithen verstreut. Die Softwarebestandteile würden organisch wachsen an den Stellen, wo sie gerade gebraucht werden.\\
\textbf{S3}:  Es wird eine Schichtenarchitektur gewählt. Eine höhere Schicht darf nur auf tiefere Schichten zugreifen. Die Wartbarkeit der Software würde davon profitieren.\\
\textbf{S4}: Es werden im gesamten System nur Software-Bibliotheken, die auf allen Betriebsystemen launfen und von jedem relevanten Browser unterstützt werden.\\
\textbf{S5}: Die gesamte Kommunikation zwischen Client und Server wird verschlüsselt. Zugriffe auf den Datenbestand erfordern die Prüfung der Nutzerberechtigung.\\
\textbf{S6}: Die Klassen des Softwaresystems werden in sinnige Module gekapselt, um eine Modularisierung zu erreichen.
\\ \hline
	      \textit{Gewählte Strategien:} \\ 
Die gewählte Strategie ist \textbf{S1}, weil der vorhandene SWP2-Prototyp das Model-View-Controller Entwurfsmuster bereits implementiert und uns dadurch Zeit spart. Wir haben mit dem SWP2-Prototypen bereits 
Erfahrungen gesammelt und ein Umstieg auf die Schichtenarchitektur wäre eine Umgewöhnung. Das Model-View-Controller Entwurfsmuster ist für die Modularisierung und die dadurch mögliche Arbeitsteilung ebenfalls gut. Außerdem wird \textbf{S4} genutzt, um zu gewährleisten, dass das System auf allen unterstützen Systemen ohne Probleme läuft.
Die Strategie \textbf{S5} wird vorerst nicht gewählt, da die Software aktuell auf lokalen Servern läuft. Im Produktiveinsatz sollte man jedoch HTTPS benutzen.  Die \textbf{S6} wird gewählt, da es die Arbeitsteilung und Gliederung der Software erleichtert.
\\ \hline
        \end{tabular}
\end{minipage}
\end{center}

% Strategie 2
\begin{center}
\begin{minipage}{\linewidth}
    \centering
\renewcommand{\arraystretch}{1.5}
\begin{tabular}{|>{\centering\arraybackslash}p{15cm}|}
            \hline
           \textbf{K2 Historisierung der Nutzeraktionen}\\ \hline
            \label{K2}
            \textit{Die Interaktionen des Nutzer sollen historisiert werden.}\\ \hline
            \textit{Betroffene Faktoren:} \\ 
	P1.2 User-Verwaltung, P1.5 Termin-Verwaltung, P2.1 Usability, O1.2 Projektbudget, O2.1 Zielsetzung, O2.2 Geringe Erfahrung, T2.1 Einbindung einer Datenbank, T2.2 Datenschutz, T2.3 Datensicherheit, T4.1 Wartbarkeit
	\\ \hline
	      \textit{Strategien:} \\
	\textbf{S7}: Es wird ein Observer-Entwurfsmuster implementiert und jede Aktion wird aufgezeichnet. Die Aufzeichnung erfolgt in einer Logging Datei.\\
	\textbf{S8}: Es wird Log4j genutzt. Diese Bibliothek bietet die Logging-Funktionalität an. Im Internet sind viele Quellen und Tutorials zu Log4j verfügbar.
\\ \hline
	      \textit{Gewählte Strategien:} \\
Es wird \textbf{S8} gewählt, da es der Weg des geringsten Widerstands ist. Die Implementierung von Log4j ist eingacher als das Schreiben eines eigenen Logging-Moduls. Außerdem erlauben die vorhandenen Quellen zu Log4j einen leichteren Einstieg in das Thema.
 \\ \hline
        \end{tabular}
\end{minipage}
\end{center}



% Strategie 3
\begin{center}
\begin{minipage}{\linewidth}
    \centering
\renewcommand{\arraystretch}{1.5}
\begin{tabular}{|>{\centering\arraybackslash}p{15cm}|}
            \hline
           \textbf{K3 Benutzung auf mobilen Geräten}\\ \hline
            \label{K3}
            \textit{Der Prüfling soll die Software auch auf seinen mobilen Geräte nutzen können.}\\ \hline
            \textit{Betroffene Faktoren:} \\ 
	  O1.1 Time-To-Market, O1.2 Projektbudget, O2.2 Mangelnde Erfahrung, P2.1 Usability, T1.1 Unterstützte Browser, T1.2 Unterstütze Betriebssysteme, T3.1 Programmiersprache, T3.2 Client-Server-Struktur, T4.1 Wartbarkeit
	\\ \hline
	      \textit{Strategien:} \\ 
	\textbf{S9}: Es wird ein eigener Client für die mobilen Geräte von uns programmiert. Die App würde eine Verbindung mit dem Server aufbauen. Der Nutzer müsste die App vor der Nutzung herunterladen.\\ 
	\textbf{S10}: Die Software auf dem Server identifiziert die verfügbare Auslösung des Cients (z.B. Chrome) und passt die angezeigte GUI darauf an. Der Nutzer geht über seinen Browser auf die Seite. Die verwendete Bibliothek wäre BootsFaces.
\\ \hline
	      \textit{Gewählte Strategien:} \\
Es wird \textbf{S10} gewählt, da der Aufwand deutlich geringer ist als eine eigene App zu entwickeln. Zu dem Thema "Responsive Design" gibt es viele Quellen für BootsFaces  im Netz. Die Wartbarkeit wäre bei der eigenen App wahrscheinlich besser, trotzdem entscheiden wir uns
angesichts der verfügbaren Zeit für die Implentierung eines responsiven Designs. Der Zugang über den Browser ist intuitiver.
\\ \hline
        \end{tabular}
\end{minipage}
\end{center}
% Strategie 4 - Die JPA und Injection Sache nochmal nachschauen.
\begin{center}
\begin{minipage}{\linewidth}
    \centering
\renewcommand{\arraystretch}{1.5}
\begin{tabular}{|>{\centering\arraybackslash}p{15cm}|}
            \hline
           \textbf{K4 Das Anbinden der Datenbank}\\ \hline
            \label{K4}
            \textit{Die Daten der Software sollen in einer Datenbank abgespeichert werden.}\\ \hline
            \textit{Betroffene Faktoren:} \\ 
	P1.2 User-Verwaltung, P1.5 Termin-Verwaltung, O1.1 Time-To-Market, O2.1 Zielsetzung, O2.2 Geringe Erfahrung, T2.1 Einbindung einer Datenbank, T2.2 Datenschutz, T2.3 Datensicherheit, T2.4 Daten-Backups, T4.1 Wartbarkeit
	\\ \hline
	      \textit{Strategien:} \\ 
	\textbf{S11}: Es wird die Datenbank direkt angebunden. Es wird keine allgemeine Schnittstelle zur Datenbankanbindung geschaffen. Es wird JPA genutzt.\\
	\textbf{S12}: Es wird eine allgemeine Schnittstelle zur Datenbankanbindung geschaffen. Die Datenbank wird darüber angebunden.Es wird JPA genutzt.\\
	\textbf{S13}: Alle Datenbankzugriffe werden auf ihren Inhalt geprüft, um so böswillige Eingaben herauszufiltern.
\\ \hline
	      \textit{Gewählte Strategien:} \\
Es werden \textbf{S12} und \textbf{S13} gewählt, weil sonst ein Austausch der Datenbank zu einem späteren Zeitpunkt deutlich mehr Aufwand erfordern würde. Ein Austauschen der Datenbank könnte durch hohe Nutzerzahlen und modernere Technologien notwendig werden. Datensicherheit ist wichtig, daher sollten böswillige Befehle in SQL-Abfragen abgefangen werden (siehe SQL-Injection).
\\ \hline
        \end{tabular}
\end{minipage}
\end{center}

% Strategie 5
\begin{center}
\begin{minipage}{\linewidth}
    \centering
\renewcommand{\arraystretch}{1.5}
\begin{tabular}{|>{\centering\arraybackslash}p{15cm}|}
            \hline
           \textbf{K5 Mehrsprachigkeit}\\ \hline
            \label{K5}
            \textit{Die Nutzer können die Sprache des Systems anpassen.}\\ \hline
            \textit{Betroffene Faktoren:} \\ 
P1.2 User-Verwaltung, P1.3 Mehrsprachigkeit, P2.1 Usability, O1.1 Time-To-Market, O2.2 Geringe Erfahrung, T1.1 Unterstützte Browser,  T4.1 Wartbarkeit \\
\hline
	      \textit{Strategien:} \\
	\textbf{S14}: Das Programm liest die Textbausteine aus einer Sprachdatei aus und fügt sie in die GUI-Elemente ein. In der Sprachdatei können mehrere Sprachen hinterlegt werden. Je nach ausgewählter Sprache werden die entsprechenden Zeilen-Nummern  zum Auslesen angepasst, um den richtigen Textbaustein auszuwählen. \\
	\textbf{S15}: Die Texte werden zweisprachig für jedes GUI-Element hinterlegt. Je nach Sprachauswahl wird eine der beiden Varianten angezeigt. \\
	\textbf{S16}: Die zuletzt ausgewählte Sprachauswahl des Nutzers wird für ihn übernommen. Beim Einloggen wird die gewählte Sprachversion angezeigt.
\\ \hline
	      \textit{Gewählte Strategien:}\\
Es werden \textbf{S15} und \textbf{S16} gewählt. Wir wählen \textbf{S15}, weil wir nur zwei Sprachen unterstützen (deutsch, englisch). Würden wir weitere Sprachen unterstützen wäre S14 unsere Wahl, jedoch wird S15 im Prototyp bereits genutzt und unsere Erfahrung in der Implentierung einer Sprachdatei ist unzureichend. Zu Gunsten der Usability wird \textbf{S16} ebenfalls genutzt, da der Nutzer so direkt in seiner Sprache begrüßt wird. Ein andauerndes Wechseln der Sprache würde den Nutzer stören, daher wird seine letzte Sprachauswahl berücksichtigt.
 \\ \hline
        \end{tabular}
\end{minipage}
\end{center}

% Strategie 6
\begin{center}
\begin{minipage}{\linewidth}
    \centering
\renewcommand{\arraystretch}{1.5}
\begin{tabular}{|>{\centering\arraybackslash}p{15cm}|}
            \hline
           \textbf{K6 Benachrichtigung der Nutzer}\\ \hline
            \label{K6}
            \textit{Die Nutzer können sich Beanchrichtigungen schicken. Die Benachrichtigung gehen an die E-Mailadresse des Nutzers. Nutzer werden auch bei wichtigen Ereignissen informiert.}\\ \hline
            \textit{Betroffene Faktoren:} \\ 
	P1.1 Einbindung des Google Kalendars, P1.2 User-Verwaltung, P1.4 Applikation für mobile Geräte, P1.5 Termin-Verwaltung, P2.1 Usability,O1.1 Time-To-Market, O2.2 Gerine Erfahrung
\\ \hline
	      \textit{Strategien:} \\
	\textbf{S17}: Es wird die JavaMail API zur Implementierung des Benachrichtigungssystems genutzt. Der Nutzer wird per E-Mail über wichtige Ereignisse informiert. Nutzer können andere Nutzer anschreiben. Die Kommunikation würde über E-Mails laufen. \\
	\textbf{S18}:  Es wird ein Nachrichtensystem ähnlich Stud.IP für die Nutzer implementiert. Die Nutzer werden im System und per E-Mail benachrichtigt. 
\\ \hline
	      \textit{Gewählte Strategien:} \\ 
Es wird Strategie \textbf{S17} ausgewählt, weil die Nutzung der API den Programmieraufwand deutlich verringern würde, auch wäre die Nutzung einer existierenden Lösung weniger Fehleranfällig. Das Benachrichtigen der Nutzer per E-Mail erfüllt die Kundenanforderungen. 
\\ \hline
        \end{tabular}
\end{minipage}
\end{center}

% Strategie 7
\begin{center}
\begin{minipage}{\linewidth}
    \centering
\renewcommand{\arraystretch}{1.5}
\begin{tabular}{|>{\centering\arraybackslash}p{15cm}|}
            \hline
           \textbf{K7 Nutzer-Verwaltung}\\ \hline
            \label{K7}
            \textit{Das System kann von mehreren Nutzern gleichzeitig genutzt werden. Jeder Nutzer hat seine eigene Session, welche abgeschottet von anderen Sessions ist, d.h. der Nutzer kann nur seine Veranstaltungen einsehen und bearbeiten.}\\ \hline
            \textit{Betroffene Faktoren:} \\
P1.1 Einbing des Google Kalendars, P1.2 User-Verwaltung, P1.4 Applikation für mobile Geräte, P1.5 Termin-Verwaltung,  O2.2 Geringe Erfahrung, T2.1 Einbindung einer Datenbank, T2.2 Datenschutz, T2.3 Datensicherheit 
 \\ \hline
	      \textit{Strategien:} \\
\textbf{S19}: Jeder Benutzer hat ein eigenes Konto. In diesem Konto sind alle Daten des Nutzer hinterlegt. Nur der Nutzer und Beteiligte (z.B. Prüfer) haben Einsicht in die Noten und Termine des Nutzers.\\
\textbf{S20}: Die  Datenbank unterstützt das ACID-Prinzip bei Datenbanktransaktionen. Die Datenbank gewährleistet damit Atomarität, Konsistenz, Isolation und Dauerhaftigkeit.\\
\textbf{S21}: Wenn ein Nutzer sich einloggt, erhält er eigene Session. Mit dem Ausloggen des Nutzer oder dem Auslaufen der Session wird seine Session beendet. Dateneinsicht und Datenänderungen sind nur innerhalb einer Session gültig.\\
\textbf{S22}: Alle nutzerbezogenen Dateien werden verschlüsselt (z.B. durch bcrypt).
\\  \hline
	      \textit{Gewählte Strategien:} \\ 
Die gewählten Strategien sind \textbf{S19}, \textbf{S20}, \textbf{S21} und \textbf{S22}. Mit \textbf{S19} wird ein Standard in der Benutzerverwaltung implementiert. Die Nutzer können ihre Konten mit einem eigenen Passwort sichern und damit ihre persönlichen Daten schützen. \textbf{S20} und \textbf{S21} sorgen für die Datenkonsistenz. Es werden nur geregelte Zugriffe auf das System zugelassen. Dadurch wollen wir die Nutzer-Verwaltung sicher gestalten. Um dieses zu gewährleisten wird \textbf{S22} genutzt, wodurch die nutzerbezogenen Daten geschützt werden.
 \\ \hline
        \end{tabular}
\end{minipage}
\end{center}

% Strategie 8
\begin{center}
\begin{minipage}{\linewidth}
    \centering
\renewcommand{\arraystretch}{1.5}
\begin{tabular}{|>{\centering\arraybackslash}p{15cm}|}
            \hline
           \textbf{K8 Daten-Backup erzeugen}\\ \hline
            \label{K8}
            \textit{Es ist möglich ein Backup vom Datenbestand zur Erstellen.}\\ \hline
            \textit{Betroffene Faktoren:} \\ 
P1.1 Time-To-Market, P1.2 User-Verwaltung, P2.1 Usability O2.2 Geringe Erfahrung, T1.2 Unterstützte Betriebssysteme, T2.1 Einbindung einer Datenbank, T2.2 Datenschutz, T2.3 Datensicherheit, T2.4 Daten-Backups, T4.1 Wartbarkeit
\\ \hline
	      \textit{Strategien:} \\ 
	\textbf{S23}: Es wird eine Komponente geschrieben, die die gesamte Datenbank einliest und in einer CSV-Datei abspeichert. Die CSV-Datei und ihre Daten werden verschlüsselt. \\
	\textbf{S24}: Es wird Daten-Backup mit Postgresql erzeugt. Das Daten-Backup wird verschlüsselt hinterlegt.
\\ \hline
	      \textit{Gewählte Strategien:} \\
Es wird die Strategie \textbf{S24} gewählt, da es vom Programmieraufwand geringer ist. Die Lösung ist bereits in Postgresql vorhanden. Außerdem existiert eine Dokumentation der Lösung.
\\ \hline
        \end{tabular}
\end{minipage}
\end{center}

% Strategie 9
\begin{center}
\begin{minipage}{\linewidth}
    \centering
\renewcommand{\arraystretch}{1.5}
\begin{tabular}{|>{\centering\arraybackslash}p{15cm}|}
            \hline
           \textbf{K9 Aktualisierung der Inhalte}\\ \hline
            \label{K9}
            \textit{Die Inhalte des Softwaresystems müssen aktualisiert werden.}\\ \hline
            \textit{Betroffene Faktoren:} \\ 
P1.1 Time-To-Market,O1.3 Team-Kapazität, P1.2 User-Verwaltung, P1.4 Applikation für mobile Geräte, P1.5 Termin-Verwaltung, P2.1 Usability O2.2 Geringe Erfahrung, T2.1 Einbindung einer Datenbank, T2.2 Datenschutz, T2.3 Datensicherheit, T3.2 Client-Server-Struktur, T4.1 Wartbarkeit
\\ \hline
	      \textit{Strategien:} \\ 
	\textbf{S25}: Das System aktualisiert sich sobald der Datenbestand verändert wurde, daher wird immer der aktuelle Datenbestand angezeigt. \\
	\textbf{S26}: Es muss ein Update-Request vom Nutzer gesendet werden, um den aktuellen Datenbestand zu sehen. Die View bei jeder Nutzeraktion aktualisiert.\\
	\textbf{S27}: Bei paralleler Nutzung wird die zeitlich erste Änderung übernommen (z.B. Eintragung in Prüfungstermin).
\\ \hline
	      \textit{Gewählte Strategien:} \\
Es wird die Strategie \textbf{S26} gewählt, da es keinen weiteren Programmieraufwand erfordert. Die Lösung ist bereits im Protorypen verbaut. Zwar wäre \textbf{S25} die schönere Lösung, jedoch verzichten wir mangels Erfahrung auf diese Strategie.
Die Strategie \textbf{S27} wird ebenfalls gewählt um so das Verhalten bei paralleler Nutzung zu bestimmen.
\\ \hline
        \end{tabular}
\end{minipage}
\end{center}