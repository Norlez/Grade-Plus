\subsection{Was ist Gradeplus?} %ReviewReady
In diesem Dokument wird die Architektur des Prüfungsverwaltungssystem \glqq{}Grade+\grqq{} 
beschrieben. Das Prüfungsverwaltungssystem \glqq{}Grade+\grqq{}  wird im Rahmen der Veranstaltung Software-Projekt 2 2017/18 entwickelt. Es handelt sich bei dem Prüfungsverwaltungssystem \glqq{}Grade+\grqq{}  um eine Weblösung in der Prüfungstermine organisiert und verwaltet werden können. 

\subsection{Rollen}
Nutzer in dem System können drei veschiedenen Rollen besitzen.\\
\\
\textbf{Prüfer:} erstellt und leitet Lehrveranstaltungen und Prüfungstermine in dem Softwaresystem. \\ 
\textbf{Prüfling:} nimmt an Prüfungsterminen teil und wird in der gewählten ILV geprüft und benotet. \\ 
\textbf{Admin:} verwaltet den Nutzer- und Datenbestand. Ist für das Anlegen von Back-Ups und Prüfern verantwortlich. \\
\\
Im folgenden werden Anwendungsbeipsiele und -anleitungen getrennt nach den Rollen beschrieben.

% Definitionen, Akronyme und Abkürzungen
%%%%%%%%%%%%%%%%%%%%%%%%%%%%%%%%%%%%%%%%
\subsection{Definitionen, Akronyme und Abkürzungen}
\textbf{Pabo}:  Das Prüfungsamt Bremen Online ist für die elektronische Erfassung der Prüfungsnoten an der Universität Bremen zuständig.\\
%\textbf{Responsive-Design:}  gestalterisches und technische Denkweise zur Erstellung von Websites, so dass diese auf Eigenschaften des jeweils benutzten Endgeräts, vor allem Smartphones reagieren können.\\
\textbf{LV}(Lehrveranstaltung): Ein grobes Schema für eine angebotene Lehrveranstaltung, die als Vorlage für konkrete Ausprägungen genutzt werden kann. Ist editierbar und erstellbar durch den Prüfer. \\
\textbf{ILV} (Instanz einer Lehrveranstaltung): Eine konkrete Ausprägung (Instanz) einer Lehrveranstaltung. Prüflinge können sich für die ILV anmelden und ihren Prüfungstermin auswählen. Prüfer können Prüfungstermine anlegen und verwalten in einer ILV. \\
%\textbf{Scope:} Die Lebenszeit eines Objekts kann in JSF durch Scopes angegeben werden, z.B. Session, falls alle Objekte der View nach einer Session zerstört werden sollen.\\
%\textbf{JPA:} Die Java Persistence API (JPA) ist eine Schnittstelle für Java-Anwendungen, die die Datenbankzugriffe und die objektrelationale Zuordnung vereinfacht.

% Dokumentenübersicht
\subsection{Übersicht über das Dokument}
Dieses Dokument ist folgendermaßen gegliedert:\\
\\
% Kapitel 1
\textbf{Kapitel \ref{sec:einführung}: Einführung} %ReviewReady
\\
\\
Das erste Kapitel erläutert den Zweck dieses Dokuments, erläuternt grundlegende Begriffe und gibt einen Überblick über das System.
\\
\\

% Kapitel 2
\textbf{Kapitel \ref{sec:Installation und Registrierung}: Installation und Registrierung} %ReviewReady
\\
\\
{ In diesem Kapitel wird die Installation von Grade+  und die Registrierung verschiedener Nutzer erklärt.}
\\
\\

% Kapitel 3
\textbf{Kapitel \ref{sec:admin}: Admin} %ReviewReady
\\
\\
{In dem diesem Kapitel werden die Aufgaben und Funktionen eines Nutzers mit der Rolle Admin aufgeführt.}
\\
\\

% Kapitel 4
\textbf{Kapitel \ref{sec:prüfer}: Prüfer} %ReviewReady
\\
\\
{In dem diesem Kapitel werden die Aufgaben und Funktionen eines Nutzers mit der Rolle Prüfer aufgeführt.}
\\
\\

% Kapitel 5
\textbf{Kapitel \ref{sec:prüflinge}: Prüfling} %ReviewReady
\\
\\
{In dem diesem Kapitel werden die Aufgaben und Funktionen eines Nutzers mit der Rolle Prüfling aufgeführt.}
\\
\\
\subsection{Instruktionen für Problemberichte}
Grade+ ist ein neuses System.
Trotz unserer sorgfälltigen Überprüfungen, können auch uns Fehler entgehen.
Jegliche Art von Kritik und vor allem das Melden gefundener Fehler ist bei den Mitarbeitern von Grade+ sehr willkommen.
Falls Sie bei der Nutzung von Grade+ Fehler finden oder ... bitte schicken Sie eine E-Mail an den Admin, der dann Grade+ direkt kontaktieren kann.