% SWP-Präambel
% C 2003-2017 Sebastian Offermann, Rainer Koschke, Karsten Hölscher
% In Zeilen 40 und 41 sind jeweils die aktuellen Daten einzutragen

\usepackage[utf8]{inputenc}     % Kodierung der Tex-Datei
\usepackage[T1]{fontenc}        % Korrekte Ausgabe von Sonderzeichen (Umlaute)
\usepackage[ngerman]{babel}     % Deutsche Einstellungen [ab \begin{document}]

\usepackage{bibgerm}            % Bibliographie
\usepackage{fancyhdr}           % obere Seitenränder gestalten
\usepackage{float}              % Floats Objekte mit [H] festsetzen
\usepackage{graphicx}           % Graphiken als jpg, png etc. einbinden
\usepackage{moreverb}           % zusätzliche verbatim-Umgebungen
\usepackage{pdflscape}          % PDF-Support für landscape
\usepackage[final]{pdfpages}    % Externe PDFs einbinden
\usepackage{stmaryrd}           % zusätzliche Symbole
\usepackage{supertabular}       % Tabellen über Seitenränder hinaus
\usepackage{tabularx}           % Tabellen mit vorgegebener Breite
\usepackage{url}                % setzt URLs schön mit \url{http://bla.laber.com/~mypage}
\usepackage{lscape}

%%% Die Reihenfolge der folgenden Pakete muss beibehalten werden:
%%% varioref, hyperref, cleveref, bookmark
% Verweise innerhalb des Dokuments schick mit " ... auf Seite ... "
% automatisch versehen. Dazu \vref{labelname} benutzen
\usepackage[ngerman]{varioref}  % [vor hyperref für korrekte Verweise]
\usepackage[colorlinks=true, pdfstartview=FitV, linkcolor=blue,
            citecolor=blue, urlcolor=blue, hyperfigures=true,
            pdftex=true]{hyperref} % [vor bookmark wegen der Optionen]
\usepackage[ngerman]{cleveref}
\usepackage{bookmark}

\hyphenation{Arbeits-paket}     % Trennungsregeln

%%% Definitionen
\newcommand{\grad}{\ensuremath{^{\circ}} }
\renewcommand{\strut}{\vrule width 0pt height5mm depth2mm}
\newcommand{\gq}[1]{\glqq{}#1\grqq{}}
\newcommand{\skipInput}[1]{}

%%% Semesterkonstanten
\newboolean{langversion} %Deklaration
\setboolean{langversion}{false} %Zuweisung ist 'false' für Blockkurs
\newcommand{\jahr}[0]{2017} %2017/2018

% erstes Argument: SWP-2, zweites SWP-1
\newcommand{\highlight}[1]{\textcolor{blue}{\textbf{#1}}}
\newcommand{\variante}[2]{\ifthenelse{\boolean{langversion}}{#1}{#2}}
\newcommand{\nurlangversion}[0]%
    {\variante{\highlight{Muss in SWP-2 ausgefüllt werden}}%
              {\highlight{Entfällt in SWP-1}}}
\newcommand{\swp}[0]{Software-Projekt \variante{2}{1}}
\newcommand{\semester}[0]{\variante{WiSe}{SoSe} \jahr}

%%% Formatierungsanpassungen
% Damit Latex nicht zu lange Zeilen produziert:
\sloppy
%Uneinheitlicher unterer Seitenrand:
%\raggedbottom

% Kein Erstzeileneinzug beim Absatzanfang
% Sieht aber nur gut aus, wenn man zwischen Absätzen viel Platz einbaut
\setlength{\parindent}{0ex}

% Abstand zwischen zwei Absätzen
\setlength{\parskip}{1ex}

% Seitenränder für Korrekturen verändern
\addtolength{\evensidemargin}{-1cm}
\addtolength{\oddsidemargin}{1cm}

\bibliographystyle{gerapali}

\newcommand\documentTitle{Bitte documentTitle festlegen!}
\newcommand\groupName{Bitte groupName festlegen!}
% swpdocument verwendet die Werte von documentTitle und groupName,
% entsprechend sollten diese vorher umgesetzt werden; sonst wird eine
% Erinnerungsmeldung an der entsprechenden Stelle im Dokument platziert
% 1. Parameter: Euer/Eure TutorIn, z. B. {Kim Harrison}
% 2. Parameter: Abgabedatum, z. B. {05. April 2063}
% 3. Parameter: Versionsnummer, z. B. {1.1}
% 4.-9. Parameter: jeweils Name und (Uni-)Email-Adresse jedes 
%                 Gruppenmitglieds; mit einem & getrennt, z. B.
% {Robin Cowl & roco@tzi.de}
% Besteht die Gruppe aus weniger als 6 Personen, so werden die 
% übrigen Parameter leer gelassen: {}
\newcommand \swpdocument[9] {
% Lustige Header auf den Seiten
  \pagestyle{fancy}
  \setlength{\headheight}{70.55003pt}
  \fancyhead{}
  \fancyhead[LO,RE]{\swp{}\\%
                    \semester{}\\%
                    \documentTitle}
  \fancyhead[LE,RO]{Seite \thepage\\%
                    \slshape \leftmark\\%
                    \slshape \rightmark}

% Lustige Header nur auf dieser Seite (Titelseite)
  \thispagestyle{fancy}
  \fancyhead[LO,RE]{ }
  \fancyhead[LE,RO]{Universität Bremen\\%
                    FB 3 -- Informatik\\%
                    Prof. Dr. Rainer Koschke\\%
                    TutorIn: #1}
  \fancyfoot[C]{}

% Start Titelseite
  \vspace{3cm}
  \begin{minipage}[H]{\textwidth}
    \begin{center}
      \bfseries \Large \swp{} -- \semester{}\\
      \smallskip
      \small VAK 03-BA-901.02\\
      \vspace{3cm}
    \end{center}
  \end{minipage}
  \begin{minipage}[H]{\textwidth}
    \begin{center}
      \vspace{1cm}
      \bfseries {\Large \documentTitle}\\
      \vspace{3ex}
      $<$\groupName$>$\\%
      \vfill
    \end{center}
  \end{minipage}
  \vfill
  \begin{minipage}[H]{\textwidth}
    \begin{center}
      \sffamily
      \begin{tabular}{lr}
        #4 \\
        #5 \\
        #6 \\
        #7 \\
        #8 \\
        #9 \\
      \end{tabular}
      \\[22mm]
      \itshape Abgabe: #2 --- Version #3 \\ ~
    \end{center}
  \end{minipage}
% Ende Titelseite

% Start Inhaltsverzeichnis
\newpage
  \thispagestyle{fancy}
  \fancyhead{}
  \fancyhead[LO,RE]{\swp{}\\%
                    \semester{}\\%
                    \documentTitle}
  \fancyhead[LE,RO]{Seite \thepage\\%
                    \slshape \leftmark\\~}
  \fancyfoot{}
  \renewcommand{\headrulewidth}{0.4pt}
  \tableofcontents
% Ende Inhaltsverzeichnis

% Header für alle weiteren Seiten
\newpage
  \fancyhead[LE,RO]{Seite \thepage\\%
                    \slshape \leftmark\\%
                    \slshape \rightmark}

}

